\documentclass{article}
\usepackage{hyperref}
\usepackage{graphicx}
\usepackage{amsmath}
\usepackage[a4paper, total={7.4in, 10in}]{geometry}

\begin{titlepage}
  \title{Report Third Project IAJ}
  \author{João Vítor ist199246
  \and Sebastião Carvalho ist199326
  \and Tiago Antunes ist199331}
  \date{2023-11-2}
\end{titlepage}

\begin{document}
  \maketitle
  \tableofcontents
  \newpage
  \section{Introduction}
  The goal of the project was to change the game of the second project, and implement a Reinforcement Learning algorithm to play it.\\
  For this, we first started with making the game restart when the player dies or wins the game, so that we can leave the agent to train
  without having to restart the simulation each time.\\
  We implemented 2 Reinforcement Learning algorithms: Q-Learning and a Neural Network.\\
  \section{Q-Learning}
  \subsection{Algorithm}
  Q-Learning is a Reinforcement Learning algorithm, this means that occasionally the agent has rewards or punishments in result of
  performing actions.\\
  For this algorithm we use macro states, these represent the current game state in a more simplified way, reducing the total
  amount of possible states for the game. Anything that is not encoded in these states cannot be learned, so these changed as
  we improved the algorithm. For each state, the algorithm must also know what actions are available to it. This is because the
  algorithm will keep a Q Table that stores the quality of an action giving a certain state, this is called the Q value.

  \[Q(s,a) = (1-alpha)Q(s, a) + alpha(r + gamma * max(Q(s', a'))),\]
  \[\text{s - current state \\a - action performed\\ r - direct reward for the performed action\\ s' - new state\\ a' - best action in the new state.}\]

  Alpha and Gamma are the learning parameters.
  Alpha, is the learning rate, takes a value between 0 and 1, so how much the new Q value will change because of the new reward.
  Gamma, is how much the indirect reward will propagate backwards, takes a value between 0 and 1 as well.\\
  
  After every execution of an action in a state, we update it's respective Q value acording to the formula above. 
  For our exploration strategy, we choose ε-greedy, this means that for learning we get the action with the highest Q value for
  the state, most of the time, but we have a low chance of using a random action.\\
  
  \subsection{Training process}
  For the training process, we let the algorithm first run for about 1000 iterations.
  We would then see the amount of victories and perform 20 runs to see the win rate.
  We would also note what the agent did in the run, to see if there was any any information
  in the state that he was missing.\\
  After this we would change what we though was necessary and train the agent again.

  \subsection{1st iteration}
  // states
  // 
  // learning parameters
  // results
  // conclusions/o que melhorar

  For our states, we started by capturing what we felt was most important while still keeping the total amount of states as low as possible.
  We captured the characters' health, money, level, position and the time of the game.\\
  These stats were simplified in the following way.\\

  \subsection{States}
  \begin{table}[h!]
      \parbox{.3\linewidth}{
        \centering
        \caption{Simplified states of Health}
        \label{tab:tableA*1}
        \begin{tabular}{c|c|c}
          \textbf{Status} & \textbf{Interval (health + shield values)}\\
          \hline
          VeryLow & [0,6]\\
          Low & [6,12]\\
          OK & [12,18]\\
          Good & [18,30]\\
        \end{tabular}
      }
      \hfil
      \parbox{.3\linewidth}{
        \centering
        \caption{Simplified states of Time}
        \label{tab:tableA*2}
        \begin{tabular}{c|c|c}
          \textbf{Status} & \textbf{Interval (time values)}\\
          \hline
          Early  & [0,49]\\
          Mid & [50,99]\\
          Late & [100,150]\\
        \end{tabular}
      }
      \hfil
      \parbox{.3\linewidth}{
        \centering
        \caption{Simplified states of Money}
        \label{tab:tableA*2}
        \begin{tabular}{c|c|c}
          \textbf{Status} & \textbf{Values}\\
          \hline
          Poor & 0 and 5\\
          Mid & 10 and 15\\
          Rich & 15, 20 and 25\\
        \end{tabular}
      }
  \end{table}
  \begin{table}[h!]
    \parbox{.45\linewidth}{
      \centering
      \caption{Simplified states of Level}
      \label{tab:tableA*1}
      \begin{tabular}{c|c|c}
        \textbf{Status} & \textbf{Level}\\
        \hline
        Level2 & 2\\
        NotLevel2 & 1 and 3\\
      \end{tabular}
    }
    \hfil
    \parbox{.45\linewidth}{
      \centering
      \caption{Simplified states of Position}
      \label{tab:tableA*2}
      \begin{tabular}{c|c|c}
        \textbf{Status} & \textbf{Interval (x coordinates)} & \textbf{Interval (z coordinates)}\\
        \hline
        TopLeft  & <=66 && >=48.27\\
        TopRight & >66 && >=48.27\\
        BottomLeft & <=66 && <48.27\\
        BottomRight & >66 && <48.27\\
      \end{tabular}
    }
  \end{table}

  This gave us a total amount of 288 possible states.\\
  The reason as to why we choose these divisions was due to our previous experiences with the game.\\
  Health is divided into intervals of 6 because that's the attack damage of the orcs, VeryLow will result in death if the
  character is attacked by an orc independently of his effective health for example, we found these values to be descriptive enough that we
  didn't need to do in intervals of the skeletons health.\\
  In terms of Time we considered 3 intervals, (I don't really remember the logic behing these 3 intervals)\\
  For money, we gave it 3 states, we represent each 2 values in a state essencially, 
  we did this in the hopes that the agent wouldn't need these intermediate values to learn that he should pick up the chests.\\
  Level, only has 2 states, this is because out of observation we can see that it is possible to win only by being in level 2, and we don't want
  to waste any more time than we need leveling up or killing enemies.\\
  The last statistic is position, in here we divided the level into rectangles, this was done by analyzing the map and seeing what areas where more
  connected so that the character would prefer actions in one area.\\

  \subsection{Reward Function}

  Our initial reward function only rewarded the agent for reaching the win condition, in which case the reward was 100, or any of the loss conditions,
  where the reward was -100.\\

  \subsection{Learning Parameters}

  At this point we didn't have a clear idea of what values where better for the learning parameters for this situation, so we chose an alpha of 0.5 and a gamma of 0.1.
  The random action chance was 5percent.\\

  \subsection{Results}

  Not very good, the guy did not win a single time :/\\

  \subsection{Conclusions}

  We must really improve.



  \subsection{2nd iteration}

  \subsection{3rd iteration}

  \subsection{4th iteration}

  \subsection{5th iteration}


  \section{Conclusions}
  

\end{document}
