\documentclass{article}
\usepackage{hyperref}
\usepackage{graphicx}
\usepackage{amsmath}
\usepackage[a4paper, total={7.4in, 10in}]{geometry}

\begin{titlepage}
  \title{Report Second Project IAJ}
  \author{João Vítor ist199246
  \and Sebastião Carvalho ist199326
  \and Tiago Antunes ist199331}
  \date{2023-10-14}
\end{titlepage}

\begin{document}
  \maketitle
  \tableofcontents
  \newpage
  \section{Introduction}
  The goal of the project was to test different decison making algortihms, and compare their performace and efficiency in terms of win rate. \\
  For the enemies we used Behaviour Trees, having a more basic one for the Mighty Dragon and the Skeletons, and a more advanced one for the Orcs, 
  which we will describe better later.\\
  For the player character, we compared 5 different algorithms: Goal Oriented Behaviour (GOB), Goal Oriented Action Planning (GOAP), Monte Carlo Search Tree (MCTS),
  MCTS with Biased Playground, and MCTS with Biased Playground and Limited Playout.
  \section{Orc's Behaviour Trees (Including Bonus Level)}
  \subsection{Basic Idea}
  For the Orcs, we needed to implement a more complex Behaviour Tree that made the Orcs patrol around 2 points, and when one saw the player, they should alert the other Orcs
  using a shout and pursue the player. The Orcs, also, need to be listening for other Orcs' shouts, and when they hear one, they move to the position where that shout occured. \\
  To do this, we designed a Behaviour Tree using Parallel Tasks and Interruptors, that allowed the Orcs to stop the patrol movement whenever they heard a shout or 
  saw the player. Also if an Orc is moving towards a shout position and sees the player, he should pursue the player and not keep moving to the shout position.\\
  \subsection{Schema}
  The schema of the Behaviour Tree is the following:
  \begin{figure}[h]
    \centering
    \includegraphics[width=0.75\textwidth ]{Schema.jpg}
  \end{figure}
  \section{GOB}
  \subsection{Algorithm}
  GOB with an overall utility function is an algorithm that uses Goals and a discontentment function, which it tries to minimize in other to find the action that better fullfills the Goals. \\
  The discontentment function used was \[discontentment = \sum_{i=1}^{k}w_i * insistence_i, \text{for k Goals}\]\\ 
  \subsection{Data}
  \begin{table}[h!]
    \centering
    \caption{GOB performance}
    \label{tab:tableGOB1}
    \begin{tabular}{c|c|c}
      \textbf{Processing time (of 1st decision)} & \textbf{Number of iterations} & \textbf{Win Rate}\\
      \hline
      1 & 1 & 1
    \end{tabular}
  \end{table}
  \subsection{Initial Analysis}
  Looking at the data, we can see GOB is a very basic algorithm but shows very good performance if the goal's weights, change rates and initial insistences are well tuned.\\
  This makes it dependent of the adjustment of the weigths according to the initial position, and only shows good win rate because there was an adjustment of the weights
  until Sir Uthgard won. 
  \section{GOAP}
  \subsection{Algorithm}
  This algorithm tries to use the idea of GOB, but applying it to sequences of actions, instead of using only one action. For this, we use a WorldState representation 
  and Depth-Limited search to find the best sequence.\\
  For our specific case, we had to use some modifications, like pruning the actions' tree of branches that had actions leading to death. The algorithm chose actions like killing
  an enemy and recovering health later, which isn't allowed on the game, so this pruning was needed.
  \subsection{Data}
  \begin{table}[h!]
    \centering
    \caption{GOAP performance}
    \label{tab:tableGOAP1}
    \begin{tabular}{c|c|c}
      \textbf{Processing time (of 1st decision)} & \textbf{Number of iterations} & \textbf{Win Rate}\\
      \hline
      1 & 1 & 1
    \end{tabular}
  \end{table}
  \subsection{Comparison}
  Comparing GOB and GOAP, we can see that GOB shows better performance, both in win rate and processing time. \\
  Since GOAP computes sequences of actions and not only a single action, it's expected to take more processing time. The win rate being lower is less clear at first, but
  ...
  
  \section{MCTS}
  \subsection{Algorithm}
  MCTS is an algorithm that was created as an alternative to the Minimax algorithm. It's basic implementation combines breadth-first tree search with local search using
  random sampling, in order to have data about if a state leads or not to a winning situation.\\
  It uses 4 steps: Selection, Expansion, Playout and Backpropagation. Selection serves to traverse the tree, selecting the optimal nodes, until a terminal node or a node that 
  still has child nodes to expand is reached. Expand, will then create that child node. In our implementation, we expand only one child node per iteration. Then, we perform 
  a Playout from this node. This consists in choosing random actions until we reach a terminal state, and then determine the result.
  In the end, all the nodes in the path are updated with this result, through Backpropagation.
  
  \subsection{Data (Next Page)}
  \begin{table}[h!]
    \centering
    \caption{MCTS performance}
    \label{tab:tableMCTS1}
    \begin{tabular}{c|c|c}
      \textbf{Processing time (of 1st decision)} & \textbf{Number of iterations} & \textbf{Win Rate}\\
      \hline
      1 & 1 & 1
    \end{tabular}
  \end{table}

  \subsection{Comparison}
  By looking at the data, we can see that the basic implementation of MCTS doesn't get us very far. This is mostly due to the randomness of the algorithm, 
  since every time we see an enemy or get close enough to a chest or potion, the algorithm runs again, giving a new decision and wasting all 
  the time spent to reach the previous target.
  Add more conclusions.\\

  \section{MCTS with Biased Playout}
  \subsection{Algorithm}
  By using an heurisitc to guide the Playout phase of MCTS, this algorithm achieves better results than the basic version.\\
  Each action gets an H value assigned, based on their class and the current world state, and then we use Gibbs distribution to sample the actions and get a probability to choose them. 
  Since we use Gibbs distribution, lower H values mean bigger probabilities.\\
  \[P(s,a_i) = \frac{e^{-h(s, a_i)}}{\sum_{j=1}^{A}e^{-h(s, a_i)}}\]
  
  \subsection{Data}
  \begin{table}[h!]
    \centering
    \caption{MCTS with Biased Playout performance}
    \label{tab:tableBiasedMCTS1}
    \begin{tabular}{c|c|c}
      \textbf{Processing time (of 1st decision)} & \textbf{Number of iterations} & \textbf{Win Rate}\\
      \hline
      1 & 1 & 1
    \end{tabular}
  \end{table}

  \subsection{Comparison}
  Comparing this data with the previous ones, we can see it has better performance in time than the basic MCTS. This is due to having less Playout iterations,
  since using a bias "guides" the playout, and there shouldn't be much change if we do multiple playouts.\\
  Comparing win rates,

  \section{MCTS with Biased Playout and Limited Playout}

  \subsection{Algorithm}
  This algorithm uses the previous as a base, but limits the maximum depth of a Playout. This is so the algorithm spends less time performing playouts. However,
  a drawback is that we may not reach a terminal branch, and in that case, we can't determine the result. To address this issue, we create an heuristic function,
  which gives a score, between 0 and 1, to a world state, that represents how good that state is, or in other words, how likely it is to lead to victory.
  
  \subsection{Data}
  \begin{table}[h!]
    \centering
    \caption{MCTS with Biased Playout and Limited Playout performance}
    \label{tab:tableLimitedBiasedMCTS1}
    \begin{tabular}{c|c|c}
      \textbf{Processing time (of 1st decision)} & \textbf{Number of iterations} & \textbf{Win Rate}\\
      \hline
      1 & 1 & 1
    \end{tabular}
  \end{table}

  \subsection{Comparison}
  We can see in the data that this algorithm has better performance then the previous versions of MCTS. This is due to the use of an heurisitc that allows us to have less
  depth in the playouts, and evaluate better the state we're in. 
  It still doesn't have a very good win rate but this is due to a tradeoff between the time it takes to make another decision and the commitment to the decision tje
  character takes. Since we don't use the tree computed on previous searches, each time we do another decision the tree resets and this can lead to another decision,
  thus wasting all the time spent in the commitment to the previous decision. Even though lowering the time to update the tree would help detecting enemies in our path,
  if we do this too often, the character just wouldn't finish any action and would still lose to time, like in the basic MCTS.\\

  \section{Conclusions}
  Analysing all algorithms we can acess that GOB is the one that shows better performance, both in win rate and processing time.\\
  Even though MCTS with Limited and Biased playout should be better theoretically, since GOB uses goals, and their weights were well adjusted to fit the 
  specific layout, it shows bigger win rate.\\
  The biggest problem with MCTS and it's variants, is that sometimes we can't detect enemies in our path, and can only detect them close to the target of our action.
  Even though we can recompute the tree when we are close to an enemy, this process makes Sir Uthgard waste all the time he already spent on the previous action if he switches
  actions. This is also not optimal, since we die not by fighting enemies, but by reaching the time limit. \\
  One optimization that we think would help is to use the previous computed tree in new searches and keep expanding it, especially in cases where we are interrupted by enemies. 
  This would make the process faster, since we wouldn't compute as much actions as before.\\
  It would also be a good improvement if in the WorldState we took account for the movement of the enemies but this would be much harder to implement, since enemies can
  change their behaviour when seeing the player, so the first optimization might be better.

\end{document}
